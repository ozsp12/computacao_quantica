\documentclass[11pt]{article}

    \usepackage[breakable]{tcolorbox}
    \usepackage{parskip} % Stop auto-indenting (to mimic markdown behaviour)
    

    % Basic figure setup, for now with no caption control since it's done
    % automatically by Pandoc (which extracts ![](path) syntax from Markdown).
    \usepackage{graphicx}
    % Maintain compatibility with old templates. Remove in nbconvert 6.0
    \let\Oldincludegraphics\includegraphics
    % Ensure that by default, figures have no caption (until we provide a
    % proper Figure object with a Caption API and a way to capture that
    % in the conversion process - todo).
    \usepackage{caption}
    \DeclareCaptionFormat{nocaption}{}
    \captionsetup{format=nocaption,aboveskip=0pt,belowskip=0pt}

    \usepackage{float}
    \floatplacement{figure}{H} % forces figures to be placed at the correct location
    \usepackage{xcolor} % Allow colors to be defined
    \usepackage{enumerate} % Needed for markdown enumerations to work
    \usepackage{geometry} % Used to adjust the document margins
    \usepackage{amsmath} % Equations
    \usepackage{amssymb} % Equations
    \usepackage{textcomp} % defines textquotesingle
    % Hack from http://tex.stackexchange.com/a/47451/13684:
    \AtBeginDocument{%
        \def\PYZsq{\textquotesingle}% Upright quotes in Pygmentized code
    }
    \usepackage{upquote} % Upright quotes for verbatim code
    \usepackage{eurosym} % defines \euro

    \usepackage{iftex}
    \ifPDFTeX
        \usepackage[T1]{fontenc}
        \IfFileExists{alphabeta.sty}{
              \usepackage{alphabeta}
          }{
              \usepackage[mathletters]{ucs}
              \usepackage[utf8x]{inputenc}
          }
    \else
        \usepackage{fontspec}
        \usepackage{unicode-math}
    \fi

    \usepackage{fancyvrb} % verbatim replacement that allows latex
    \usepackage{grffile} % extends the file name processing of package graphics 
                         % to support a larger range
    \makeatletter % fix for old versions of grffile with XeLaTeX
    \@ifpackagelater{grffile}{2019/11/01}
    {
      % Do nothing on new versions
    }
    {
      \def\Gread@@xetex#1{%
        \IfFileExists{"\Gin@base".bb}%
        {\Gread@eps{\Gin@base.bb}}%
        {\Gread@@xetex@aux#1}%
      }
    }
    \makeatother
    \usepackage[Export]{adjustbox} % Used to constrain images to a maximum size
    \adjustboxset{max size={0.9\linewidth}{0.9\paperheight}}

    % The hyperref package gives us a pdf with properly built
    % internal navigation ('pdf bookmarks' for the table of contents,
    % internal cross-reference links, web links for URLs, etc.)
    \usepackage{hyperref}
    % The default LaTeX title has an obnoxious amount of whitespace. By default,
    % titling removes some of it. It also provides customization options.
    \usepackage{titling}
    \usepackage{longtable} % longtable support required by pandoc >1.10
    \usepackage{booktabs}  % table support for pandoc > 1.12.2
    \usepackage{array}     % table support for pandoc >= 2.11.3
    \usepackage{calc}      % table minipage width calculation for pandoc >= 2.11.1
    \usepackage[inline]{enumitem} % IRkernel/repr support (it uses the enumerate* environment)
    \usepackage[normalem]{ulem} % ulem is needed to support strikethroughs (\sout)
                                % normalem makes italics be italics, not underlines
    \usepackage{mathrsfs}
    
   
    % Colors for the hyperref package
    \definecolor{urlcolor}{rgb}{0,.145,.698}
    \definecolor{linkcolor}{rgb}{.71,0.21,0.01}
    \definecolor{citecolor}{rgb}{.12,.54,.11}

    % ANSI colors
    \definecolor{ansi-black}{HTML}{3E424D}
    \definecolor{ansi-black-intense}{HTML}{282C36}
    \definecolor{ansi-red}{HTML}{E75C58}
    \definecolor{ansi-red-intense}{HTML}{B22B31}
    \definecolor{ansi-green}{HTML}{00A250}
    \definecolor{ansi-green-intense}{HTML}{007427}
    \definecolor{ansi-yellow}{HTML}{DDB62B}
    \definecolor{ansi-yellow-intense}{HTML}{B27D12}
    \definecolor{ansi-blue}{HTML}{208FFB}
    \definecolor{ansi-blue-intense}{HTML}{0065CA}
    \definecolor{ansi-magenta}{HTML}{D160C4}
    \definecolor{ansi-magenta-intense}{HTML}{A03196}
    \definecolor{ansi-cyan}{HTML}{60C6C8}
    \definecolor{ansi-cyan-intense}{HTML}{258F8F}
    \definecolor{ansi-white}{HTML}{C5C1B4}
    \definecolor{ansi-white-intense}{HTML}{A1A6B2}
    \definecolor{ansi-default-inverse-fg}{HTML}{FFFFFF}
    \definecolor{ansi-default-inverse-bg}{HTML}{000000}

    % common color for the border for error outputs.
    \definecolor{outerrorbackground}{HTML}{FFDFDF}

    % commands and environments needed by pandoc snippets
    % extracted from the output of `pandoc -s`
    \providecommand{\tightlist}{%
      \setlength{\itemsep}{0pt}\setlength{\parskip}{0pt}}
    \DefineVerbatimEnvironment{Highlighting}{Verbatim}{commandchars=\\\{\}}
    % Add ',fontsize=\small' for more characters per line
    \newenvironment{Shaded}{}{}
    \newcommand{\KeywordTok}[1]{\textcolor[rgb]{0.00,0.44,0.13}{\textbf{{#1}}}}
    \newcommand{\DataTypeTok}[1]{\textcolor[rgb]{0.56,0.13,0.00}{{#1}}}
    \newcommand{\DecValTok}[1]{\textcolor[rgb]{0.25,0.63,0.44}{{#1}}}
    \newcommand{\BaseNTok}[1]{\textcolor[rgb]{0.25,0.63,0.44}{{#1}}}
    \newcommand{\FloatTok}[1]{\textcolor[rgb]{0.25,0.63,0.44}{{#1}}}
    \newcommand{\CharTok}[1]{\textcolor[rgb]{0.25,0.44,0.63}{{#1}}}
    \newcommand{\StringTok}[1]{\textcolor[rgb]{0.25,0.44,0.63}{{#1}}}
    \newcommand{\CommentTok}[1]{\textcolor[rgb]{0.38,0.63,0.69}{\textit{{#1}}}}
    \newcommand{\OtherTok}[1]{\textcolor[rgb]{0.00,0.44,0.13}{{#1}}}
    \newcommand{\AlertTok}[1]{\textcolor[rgb]{1.00,0.00,0.00}{\textbf{{#1}}}}
    \newcommand{\FunctionTok}[1]{\textcolor[rgb]{0.02,0.16,0.49}{{#1}}}
    \newcommand{\RegionMarkerTok}[1]{{#1}}
    \newcommand{\ErrorTok}[1]{\textcolor[rgb]{1.00,0.00,0.00}{\textbf{{#1}}}}
    \newcommand{\NormalTok}[1]{{#1}}
    
    % Additional commands for more recent versions of Pandoc
    \newcommand{\ConstantTok}[1]{\textcolor[rgb]{0.53,0.00,0.00}{{#1}}}
    \newcommand{\SpecialCharTok}[1]{\textcolor[rgb]{0.25,0.44,0.63}{{#1}}}
    \newcommand{\VerbatimStringTok}[1]{\textcolor[rgb]{0.25,0.44,0.63}{{#1}}}
    \newcommand{\SpecialStringTok}[1]{\textcolor[rgb]{0.73,0.40,0.53}{{#1}}}
    \newcommand{\ImportTok}[1]{{#1}}
    \newcommand{\DocumentationTok}[1]{\textcolor[rgb]{0.73,0.13,0.13}{\textit{{#1}}}}
    \newcommand{\AnnotationTok}[1]{\textcolor[rgb]{0.38,0.63,0.69}{\textbf{\textit{{#1}}}}}
    \newcommand{\CommentVarTok}[1]{\textcolor[rgb]{0.38,0.63,0.69}{\textbf{\textit{{#1}}}}}
    \newcommand{\VariableTok}[1]{\textcolor[rgb]{0.10,0.09,0.49}{{#1}}}
    \newcommand{\ControlFlowTok}[1]{\textcolor[rgb]{0.00,0.44,0.13}{\textbf{{#1}}}}
    \newcommand{\OperatorTok}[1]{\textcolor[rgb]{0.40,0.40,0.40}{{#1}}}
    \newcommand{\BuiltInTok}[1]{{#1}}
    \newcommand{\ExtensionTok}[1]{{#1}}
    \newcommand{\PreprocessorTok}[1]{\textcolor[rgb]{0.74,0.48,0.00}{{#1}}}
    \newcommand{\AttributeTok}[1]{\textcolor[rgb]{0.49,0.56,0.16}{{#1}}}
    \newcommand{\InformationTok}[1]{\textcolor[rgb]{0.38,0.63,0.69}{\textbf{\textit{{#1}}}}}
    \newcommand{\WarningTok}[1]{\textcolor[rgb]{0.38,0.63,0.69}{\textbf{\textit{{#1}}}}}
    
    
    % Define a nice break command that doesn't care if a line doesn't already
    % exist.
    \def\br{\hspace*{\fill} \\* }
    % Math Jax compatibility definitions
    \def\gt{>}
    \def\lt{<}
    \let\Oldtex\TeX
    \let\Oldlatex\LaTeX
    \renewcommand{\TeX}{\textrm{\Oldtex}}
    \renewcommand{\LaTeX}{\textrm{\Oldlatex}}
    % Document parameters
    % Document title
    \title{exemplo\_circuito\_01}
    
    
    
    
    
% Pygments definitions
\makeatletter
\def\PY@reset{\let\PY@it=\relax \let\PY@bf=\relax%
    \let\PY@ul=\relax \let\PY@tc=\relax%
    \let\PY@bc=\relax \let\PY@ff=\relax}
\def\PY@tok#1{\csname PY@tok@#1\endcsname}
\def\PY@toks#1+{\ifx\relax#1\empty\else%
    \PY@tok{#1}\expandafter\PY@toks\fi}
\def\PY@do#1{\PY@bc{\PY@tc{\PY@ul{%
    \PY@it{\PY@bf{\PY@ff{#1}}}}}}}
\def\PY#1#2{\PY@reset\PY@toks#1+\relax+\PY@do{#2}}

\@namedef{PY@tok@w}{\def\PY@tc##1{\textcolor[rgb]{0.73,0.73,0.73}{##1}}}
\@namedef{PY@tok@c}{\let\PY@it=\textit\def\PY@tc##1{\textcolor[rgb]{0.24,0.48,0.48}{##1}}}
\@namedef{PY@tok@cp}{\def\PY@tc##1{\textcolor[rgb]{0.61,0.40,0.00}{##1}}}
\@namedef{PY@tok@k}{\let\PY@bf=\textbf\def\PY@tc##1{\textcolor[rgb]{0.00,0.50,0.00}{##1}}}
\@namedef{PY@tok@kp}{\def\PY@tc##1{\textcolor[rgb]{0.00,0.50,0.00}{##1}}}
\@namedef{PY@tok@kt}{\def\PY@tc##1{\textcolor[rgb]{0.69,0.00,0.25}{##1}}}
\@namedef{PY@tok@o}{\def\PY@tc##1{\textcolor[rgb]{0.40,0.40,0.40}{##1}}}
\@namedef{PY@tok@ow}{\let\PY@bf=\textbf\def\PY@tc##1{\textcolor[rgb]{0.67,0.13,1.00}{##1}}}
\@namedef{PY@tok@nb}{\def\PY@tc##1{\textcolor[rgb]{0.00,0.50,0.00}{##1}}}
\@namedef{PY@tok@nf}{\def\PY@tc##1{\textcolor[rgb]{0.00,0.00,1.00}{##1}}}
\@namedef{PY@tok@nc}{\let\PY@bf=\textbf\def\PY@tc##1{\textcolor[rgb]{0.00,0.00,1.00}{##1}}}
\@namedef{PY@tok@nn}{\let\PY@bf=\textbf\def\PY@tc##1{\textcolor[rgb]{0.00,0.00,1.00}{##1}}}
\@namedef{PY@tok@ne}{\let\PY@bf=\textbf\def\PY@tc##1{\textcolor[rgb]{0.80,0.25,0.22}{##1}}}
\@namedef{PY@tok@nv}{\def\PY@tc##1{\textcolor[rgb]{0.10,0.09,0.49}{##1}}}
\@namedef{PY@tok@no}{\def\PY@tc##1{\textcolor[rgb]{0.53,0.00,0.00}{##1}}}
\@namedef{PY@tok@nl}{\def\PY@tc##1{\textcolor[rgb]{0.46,0.46,0.00}{##1}}}
\@namedef{PY@tok@ni}{\let\PY@bf=\textbf\def\PY@tc##1{\textcolor[rgb]{0.44,0.44,0.44}{##1}}}
\@namedef{PY@tok@na}{\def\PY@tc##1{\textcolor[rgb]{0.41,0.47,0.13}{##1}}}
\@namedef{PY@tok@nt}{\let\PY@bf=\textbf\def\PY@tc##1{\textcolor[rgb]{0.00,0.50,0.00}{##1}}}
\@namedef{PY@tok@nd}{\def\PY@tc##1{\textcolor[rgb]{0.67,0.13,1.00}{##1}}}
\@namedef{PY@tok@s}{\def\PY@tc##1{\textcolor[rgb]{0.73,0.13,0.13}{##1}}}
\@namedef{PY@tok@sd}{\let\PY@it=\textit\def\PY@tc##1{\textcolor[rgb]{0.73,0.13,0.13}{##1}}}
\@namedef{PY@tok@si}{\let\PY@bf=\textbf\def\PY@tc##1{\textcolor[rgb]{0.64,0.35,0.47}{##1}}}
\@namedef{PY@tok@se}{\let\PY@bf=\textbf\def\PY@tc##1{\textcolor[rgb]{0.67,0.36,0.12}{##1}}}
\@namedef{PY@tok@sr}{\def\PY@tc##1{\textcolor[rgb]{0.64,0.35,0.47}{##1}}}
\@namedef{PY@tok@ss}{\def\PY@tc##1{\textcolor[rgb]{0.10,0.09,0.49}{##1}}}
\@namedef{PY@tok@sx}{\def\PY@tc##1{\textcolor[rgb]{0.00,0.50,0.00}{##1}}}
\@namedef{PY@tok@m}{\def\PY@tc##1{\textcolor[rgb]{0.40,0.40,0.40}{##1}}}
\@namedef{PY@tok@gh}{\let\PY@bf=\textbf\def\PY@tc##1{\textcolor[rgb]{0.00,0.00,0.50}{##1}}}
\@namedef{PY@tok@gu}{\let\PY@bf=\textbf\def\PY@tc##1{\textcolor[rgb]{0.50,0.00,0.50}{##1}}}
\@namedef{PY@tok@gd}{\def\PY@tc##1{\textcolor[rgb]{0.63,0.00,0.00}{##1}}}
\@namedef{PY@tok@gi}{\def\PY@tc##1{\textcolor[rgb]{0.00,0.52,0.00}{##1}}}
\@namedef{PY@tok@gr}{\def\PY@tc##1{\textcolor[rgb]{0.89,0.00,0.00}{##1}}}
\@namedef{PY@tok@ge}{\let\PY@it=\textit}
\@namedef{PY@tok@gs}{\let\PY@bf=\textbf}
\@namedef{PY@tok@ges}{\let\PY@bf=\textbf\let\PY@it=\textit}
\@namedef{PY@tok@gp}{\let\PY@bf=\textbf\def\PY@tc##1{\textcolor[rgb]{0.00,0.00,0.50}{##1}}}
\@namedef{PY@tok@go}{\def\PY@tc##1{\textcolor[rgb]{0.44,0.44,0.44}{##1}}}
\@namedef{PY@tok@gt}{\def\PY@tc##1{\textcolor[rgb]{0.00,0.27,0.87}{##1}}}
\@namedef{PY@tok@err}{\def\PY@bc##1{{\setlength{\fboxsep}{\string -\fboxrule}\fcolorbox[rgb]{1.00,0.00,0.00}{1,1,1}{\strut ##1}}}}
\@namedef{PY@tok@kc}{\let\PY@bf=\textbf\def\PY@tc##1{\textcolor[rgb]{0.00,0.50,0.00}{##1}}}
\@namedef{PY@tok@kd}{\let\PY@bf=\textbf\def\PY@tc##1{\textcolor[rgb]{0.00,0.50,0.00}{##1}}}
\@namedef{PY@tok@kn}{\let\PY@bf=\textbf\def\PY@tc##1{\textcolor[rgb]{0.00,0.50,0.00}{##1}}}
\@namedef{PY@tok@kr}{\let\PY@bf=\textbf\def\PY@tc##1{\textcolor[rgb]{0.00,0.50,0.00}{##1}}}
\@namedef{PY@tok@bp}{\def\PY@tc##1{\textcolor[rgb]{0.00,0.50,0.00}{##1}}}
\@namedef{PY@tok@fm}{\def\PY@tc##1{\textcolor[rgb]{0.00,0.00,1.00}{##1}}}
\@namedef{PY@tok@vc}{\def\PY@tc##1{\textcolor[rgb]{0.10,0.09,0.49}{##1}}}
\@namedef{PY@tok@vg}{\def\PY@tc##1{\textcolor[rgb]{0.10,0.09,0.49}{##1}}}
\@namedef{PY@tok@vi}{\def\PY@tc##1{\textcolor[rgb]{0.10,0.09,0.49}{##1}}}
\@namedef{PY@tok@vm}{\def\PY@tc##1{\textcolor[rgb]{0.10,0.09,0.49}{##1}}}
\@namedef{PY@tok@sa}{\def\PY@tc##1{\textcolor[rgb]{0.73,0.13,0.13}{##1}}}
\@namedef{PY@tok@sb}{\def\PY@tc##1{\textcolor[rgb]{0.73,0.13,0.13}{##1}}}
\@namedef{PY@tok@sc}{\def\PY@tc##1{\textcolor[rgb]{0.73,0.13,0.13}{##1}}}
\@namedef{PY@tok@dl}{\def\PY@tc##1{\textcolor[rgb]{0.73,0.13,0.13}{##1}}}
\@namedef{PY@tok@s2}{\def\PY@tc##1{\textcolor[rgb]{0.73,0.13,0.13}{##1}}}
\@namedef{PY@tok@sh}{\def\PY@tc##1{\textcolor[rgb]{0.73,0.13,0.13}{##1}}}
\@namedef{PY@tok@s1}{\def\PY@tc##1{\textcolor[rgb]{0.73,0.13,0.13}{##1}}}
\@namedef{PY@tok@mb}{\def\PY@tc##1{\textcolor[rgb]{0.40,0.40,0.40}{##1}}}
\@namedef{PY@tok@mf}{\def\PY@tc##1{\textcolor[rgb]{0.40,0.40,0.40}{##1}}}
\@namedef{PY@tok@mh}{\def\PY@tc##1{\textcolor[rgb]{0.40,0.40,0.40}{##1}}}
\@namedef{PY@tok@mi}{\def\PY@tc##1{\textcolor[rgb]{0.40,0.40,0.40}{##1}}}
\@namedef{PY@tok@il}{\def\PY@tc##1{\textcolor[rgb]{0.40,0.40,0.40}{##1}}}
\@namedef{PY@tok@mo}{\def\PY@tc##1{\textcolor[rgb]{0.40,0.40,0.40}{##1}}}
\@namedef{PY@tok@ch}{\let\PY@it=\textit\def\PY@tc##1{\textcolor[rgb]{0.24,0.48,0.48}{##1}}}
\@namedef{PY@tok@cm}{\let\PY@it=\textit\def\PY@tc##1{\textcolor[rgb]{0.24,0.48,0.48}{##1}}}
\@namedef{PY@tok@cpf}{\let\PY@it=\textit\def\PY@tc##1{\textcolor[rgb]{0.24,0.48,0.48}{##1}}}
\@namedef{PY@tok@c1}{\let\PY@it=\textit\def\PY@tc##1{\textcolor[rgb]{0.24,0.48,0.48}{##1}}}
\@namedef{PY@tok@cs}{\let\PY@it=\textit\def\PY@tc##1{\textcolor[rgb]{0.24,0.48,0.48}{##1}}}

\def\PYZbs{\char`\\}
\def\PYZus{\char`\_}
\def\PYZob{\char`\{}
\def\PYZcb{\char`\}}
\def\PYZca{\char`\^}
\def\PYZam{\char`\&}
\def\PYZlt{\char`\<}
\def\PYZgt{\char`\>}
\def\PYZsh{\char`\#}
\def\PYZpc{\char`\%}
\def\PYZdl{\char`\$}
\def\PYZhy{\char`\-}
\def\PYZsq{\char`\'}
\def\PYZdq{\char`\"}
\def\PYZti{\char`\~}
% for compatibility with earlier versions
\def\PYZat{@}
\def\PYZlb{[}
\def\PYZrb{]}
\makeatother


    % For linebreaks inside Verbatim environment from package fancyvrb. 
    \makeatletter
        \newbox\Wrappedcontinuationbox 
        \newbox\Wrappedvisiblespacebox 
        \newcommand*\Wrappedvisiblespace {\textcolor{red}{\textvisiblespace}} 
        \newcommand*\Wrappedcontinuationsymbol {\textcolor{red}{\llap{\tiny$\m@th\hookrightarrow$}}} 
        \newcommand*\Wrappedcontinuationindent {3ex } 
        \newcommand*\Wrappedafterbreak {\kern\Wrappedcontinuationindent\copy\Wrappedcontinuationbox} 
        % Take advantage of the already applied Pygments mark-up to insert 
        % potential linebreaks for TeX processing. 
        %        {, <, #, %, $, ' and ": go to next line. 
        %        _, }, ^, &, >, - and ~: stay at end of broken line. 
        % Use of \textquotesingle for straight quote. 
        \newcommand*\Wrappedbreaksatspecials {% 
            \def\PYGZus{\discretionary{\char`\_}{\Wrappedafterbreak}{\char`\_}}% 
            \def\PYGZob{\discretionary{}{\Wrappedafterbreak\char`\{}{\char`\{}}% 
            \def\PYGZcb{\discretionary{\char`\}}{\Wrappedafterbreak}{\char`\}}}% 
            \def\PYGZca{\discretionary{\char`\^}{\Wrappedafterbreak}{\char`\^}}% 
            \def\PYGZam{\discretionary{\char`\&}{\Wrappedafterbreak}{\char`\&}}% 
            \def\PYGZlt{\discretionary{}{\Wrappedafterbreak\char`\<}{\char`\<}}% 
            \def\PYGZgt{\discretionary{\char`\>}{\Wrappedafterbreak}{\char`\>}}% 
            \def\PYGZsh{\discretionary{}{\Wrappedafterbreak\char`\#}{\char`\#}}% 
            \def\PYGZpc{\discretionary{}{\Wrappedafterbreak\char`\%}{\char`\%}}% 
            \def\PYGZdl{\discretionary{}{\Wrappedafterbreak\char`\$}{\char`\$}}% 
            \def\PYGZhy{\discretionary{\char`\-}{\Wrappedafterbreak}{\char`\-}}% 
            \def\PYGZsq{\discretionary{}{\Wrappedafterbreak\textquotesingle}{\textquotesingle}}% 
            \def\PYGZdq{\discretionary{}{\Wrappedafterbreak\char`\"}{\char`\"}}% 
            \def\PYGZti{\discretionary{\char`\~}{\Wrappedafterbreak}{\char`\~}}% 
        } 
        % Some characters . , ; ? ! / are not pygmentized. 
        % This macro makes them "active" and they will insert potential linebreaks 
        \newcommand*\Wrappedbreaksatpunct {% 
            \lccode`\~`\.\lowercase{\def~}{\discretionary{\hbox{\char`\.}}{\Wrappedafterbreak}{\hbox{\char`\.}}}% 
            \lccode`\~`\,\lowercase{\def~}{\discretionary{\hbox{\char`\,}}{\Wrappedafterbreak}{\hbox{\char`\,}}}% 
            \lccode`\~`\;\lowercase{\def~}{\discretionary{\hbox{\char`\;}}{\Wrappedafterbreak}{\hbox{\char`\;}}}% 
            \lccode`\~`\:\lowercase{\def~}{\discretionary{\hbox{\char`\:}}{\Wrappedafterbreak}{\hbox{\char`\:}}}% 
            \lccode`\~`\?\lowercase{\def~}{\discretionary{\hbox{\char`\?}}{\Wrappedafterbreak}{\hbox{\char`\?}}}% 
            \lccode`\~`\!\lowercase{\def~}{\discretionary{\hbox{\char`\!}}{\Wrappedafterbreak}{\hbox{\char`\!}}}% 
            \lccode`\~`\/\lowercase{\def~}{\discretionary{\hbox{\char`\/}}{\Wrappedafterbreak}{\hbox{\char`\/}}}% 
            \catcode`\.\active
            \catcode`\,\active 
            \catcode`\;\active
            \catcode`\:\active
            \catcode`\?\active
            \catcode`\!\active
            \catcode`\/\active 
            \lccode`\~`\~ 	
        }
    \makeatother

    \let\OriginalVerbatim=\Verbatim
    \makeatletter
    \renewcommand{\Verbatim}[1][1]{%
        %\parskip\z@skip
        \sbox\Wrappedcontinuationbox {\Wrappedcontinuationsymbol}%
        \sbox\Wrappedvisiblespacebox {\FV@SetupFont\Wrappedvisiblespace}%
        \def\FancyVerbFormatLine ##1{\hsize\linewidth
            \vtop{\raggedright\hyphenpenalty\z@\exhyphenpenalty\z@
                \doublehyphendemerits\z@\finalhyphendemerits\z@
                \strut ##1\strut}%
        }%
        % If the linebreak is at a space, the latter will be displayed as visible
        % space at end of first line, and a continuation symbol starts next line.
        % Stretch/shrink are however usually zero for typewriter font.
        \def\FV@Space {%
            \nobreak\hskip\z@ plus\fontdimen3\font minus\fontdimen4\font
            \discretionary{\copy\Wrappedvisiblespacebox}{\Wrappedafterbreak}
            {\kern\fontdimen2\font}%
        }%
        
        % Allow breaks at special characters using \PYG... macros.
        \Wrappedbreaksatspecials
        % Breaks at punctuation characters . , ; ? ! and / need catcode=\active 	
        \OriginalVerbatim[#1,codes*=\Wrappedbreaksatpunct]%
    }
    \makeatother

    % Exact colors from NB
    \definecolor{incolor}{HTML}{303F9F}
    \definecolor{outcolor}{HTML}{D84315}
    \definecolor{cellborder}{HTML}{CFCFCF}
    \definecolor{cellbackground}{HTML}{F7F7F7}
    
    % prompt
    \makeatletter
    \newcommand{\boxspacing}{\kern\kvtcb@left@rule\kern\kvtcb@boxsep}
    \makeatother
    \newcommand{\prompt}[4]{
        {\ttfamily\llap{{\color{#2}[#3]:\hspace{3pt}#4}}\vspace{-\baselineskip}}
    }
    

    
    % Prevent overflowing lines due to hard-to-break entities
    \sloppy 
    % Setup hyperref package
    \hypersetup{
      breaklinks=true,  % so long urls are correctly broken across lines
      colorlinks=true,
      urlcolor=urlcolor,
      linkcolor=linkcolor,
      citecolor=citecolor,
      }
    % Slightly bigger margins than the latex defaults
    
    \geometry{verbose,tmargin=1in,bmargin=1in,lmargin=1in,rmargin=1in}
    
    

\begin{document}
    
    \maketitle  
		
    \textbf{Autor:} Osvaldo L. Santos-Pereira (olsp@if.ufrj.br /
osvald23@gmail.com)


\begin{itemize}
    \item \textbf{Webpage:} \url{https://ozsp12.github.io/}
    \item \textbf{GitHub:} \url{https://github.com/ozsp12}
    \item \textbf{ResearchGate:} \url{https://www.researchgate.net/profile/Osvaldo-Santos-Pereira}
    \item \textbf{Google Scholar:} \url{https://scholar.google.com/citations?user=HIZp0X8AAAAJ&hl=en}
    \item \textbf{ORCID:} \url{https://orcid.org/0000-0003-2231-517X}
    \item \textbf{LinkedIn:} \url{https://www.linkedin.com/in/ozsp12}
    \item \textbf{Twitter (X):} \url{https://x.com/ozsp12}
    \item \textbf{TikTok:} \url{https://www.tiktok.com/@ozsp12}
    \item \textbf{YouTube:} \url{https://www.youtube.com/@ozlsp12}
\end{itemize}
 

    \section{Introdução}\label{introduuxe7uxe3o}

\begin{itemize}
\tightlist
\item
  O qubit \(q_0\) é usado como \textbf{controle lógico} de operações
  subsequentes.\\
\item
  O qubit \(q_1\) atua como \textbf{alvo} das rotações e da operação
  CNOT.\\
\item
  As duas portas Hadamard em \(q_0\) criam e depois reconfiguram a base
  de superposição.\\
\item
  A porta \(R_Y(\pi)\) é responsável por introduzir uma rotação contínua
  (analógica) na esfera de Bloch, produzindo interferência de fase.
\end{itemize}

Em conjunto, essas operações geram um circuito capaz de
\textbf{preparar, manipular e medir estados de Bell}. O circuito atua
primeiro sobre o qubit \(q_0\), aplicando a porta \(X\), seguida da
porta \(H\), e depois um \(CNOT\) no qual \(q_0\) é o qubit de controle.
Após essa etapa de controle, ainda em \(q_0\), aplica-se uma segunda
porta \(H\). No qubit \(q_1\), que funciona como alvo do \(CNOT\),
aplica-se ao final a rotação \(R_Y(\pi)\). Em resumo, trata-se de um
circuito em que \(q_0\) controla \(q_1\), com operações adicionais que
preparam e transformam o estado antes e depois do acoplamento entre
eles.

No qubit \(q_0\) são aplicadas as seguintes portas: \\ 1. \(X\) \\ 2. \(H\) 3.
\(\bullet\!-\!H\) (controle de um CNOT seguido de um H)

No qubit \(q_1\) são aplicadas as seguintes portas:

\begin{enumerate}
\def\labelenumi{\arabic{enumi}.}
\tightlist
\item
  \(\oplus \quad \text{(alvo do CNOT)}\)
\item
  \(R_Y(\pi)\)
\end{enumerate}

Ou seja, a porta \(CNOT\) é aplicada de modo que o qubit \(q_0\)
desempenha o papel de controle, determinando se a operação é ou não
acionada, enquanto o qubit \(q_1\) atua como alvo, sofrendo a inversão
condicionada ao estado de \(q_0\). Fisicamente, dizer que um qubit
desempenha o papel de controle enquanto o outro é o alvo significa que a
evolução de um deles determina de maneira coerente o que acontece com o
outro. No caso da porta \(CNOT\), o qubit de controle não sofre nenhuma
alteração direta: ele apenas ``decide'', pelo seu estado, se o alvo será
modificado. Se o controle estiver em \(|0\rangle\), nada acontece; se
estiver em \(|1\rangle\), o alvo sofre um flip por meio da operação
\(X\). O ponto crucial é que isso ocorre sem medição: se o controle
estiver em superposição, a operação acontece de forma condicional e
simultânea em cada ramo da superposição, produzindo correlações
quânticas e gerando emaranhamento. Assim, o controle é o sistema que
aciona a interação, e o alvo é o sistema que recebe o efeito da operação
condicionada, tudo preservando a coerência e a linearidade da mecânica
quântica.

 \begin{tcolorbox}[breakable, size=fbox, boxrule=1pt, pad at break*=1mm,colback=cellbackground, colframe=cellborder]
\prompt{In}{incolor}{1}{\boxspacing}
\begin{Verbatim}[commandchars=\\\{\}]
\PY{c+c1}{\PYZsh{} Imports principais}
\PY{k+kn}{from}\PY{+w}{ }\PY{n+nn}{qiskit}\PY{+w}{ }\PY{k+kn}{import} \PY{n}{QuantumCircuit}\PY{p}{,} \PY{n}{QuantumRegister}
\PY{k+kn}{from}\PY{+w}{ }\PY{n+nn}{qiskit}\PY{n+nn}{.}\PY{n+nn}{quantum\PYZus{}info}\PY{+w}{ }\PY{k+kn}{import} \PY{n}{Statevector}
\PY{k+kn}{from}\PY{+w}{ }\PY{n+nn}{qiskit}\PY{n+nn}{.}\PY{n+nn}{visualization}\PY{+w}{ }\PY{k+kn}{import} \PY{n}{plot\PYZus{}bloch\PYZus{}multivector}
\PY{k+kn}{import}\PY{+w}{ }\PY{n+nn}{numpy}\PY{+w}{ }\PY{k}{as}\PY{+w}{ }\PY{n+nn}{np}
\PY{c+c1}{\PYZsh{} comando que faz com que os gráficos gerados usando Matplotlib sejam exibidos diretamente dentro do próprio notebook}
\PY{o}{\PYZpc{}}\PY{k}{matplotlib} inline

\PY{c+c1}{\PYZsh{} Cria um circuito com 2 qubits}
\PY{n}{qc} \PY{o}{=} \PY{n}{QuantumCircuit}\PY{p}{(}\PY{l+m+mi}{2}\PY{p}{)}
\PY{c+c1}{\PYZsh{} Aplica a porta X no qubit 0}
\PY{n}{qc}\PY{o}{.}\PY{n}{x}\PY{p}{(}\PY{l+m+mi}{0}\PY{p}{)}
\PY{c+c1}{\PYZsh{} Aplica uma porta Hadamard (H) no qubit 0}
\PY{n}{qc}\PY{o}{.}\PY{n}{h}\PY{p}{(}\PY{l+m+mi}{0}\PY{p}{)}
\PY{c+c1}{\PYZsh{} Aplica uma porta CNOT com controle em q0 e alvo em q1}
\PY{n}{qc}\PY{o}{.}\PY{n}{cx}\PY{p}{(}\PY{l+m+mi}{0}\PY{p}{,} \PY{l+m+mi}{1}\PY{p}{)}
\PY{c+c1}{\PYZsh{} Aplica uma rotação R\PYZus{}y(π) no qubit 1}
\PY{n}{qc}\PY{o}{.}\PY{n}{ry}\PY{p}{(}\PY{n}{np}\PY{o}{.}\PY{n}{pi}\PY{p}{,} \PY{l+m+mi}{1}\PY{p}{)}
\PY{c+c1}{\PYZsh{} Aplica novamente uma porta Hadamard no qubit 0}
\PY{n}{qc}\PY{o}{.}\PY{n}{h}\PY{p}{(}\PY{l+m+mi}{0}\PY{p}{)}
\PY{c+c1}{\PYZsh{} Exibe o circuito}
\PY{n}{qc}\PY{o}{.}\PY{n}{draw}\PY{p}{(}\PY{l+s+s2}{\PYZdq{}}\PY{l+s+s2}{mpl}\PY{l+s+s2}{\PYZdq{}}\PY{p}{)}
\end{Verbatim}
\end{tcolorbox}
 
            
\prompt{Out}{outcolor}{1}{}
    
    \begin{center}
    \adjustimage{max size={0.9\linewidth}{0.9\paperheight}}{exemplo_circuito_01_files/exemplo_circuito_01_2_0.png}
    \end{center}
    { \hspace*{\fill} \\}
    

    \section{Explicação passo a passo}\label{explicauxe7uxe3o-passo-a-passo}

    \subsection{Criando a registradora}\label{criando-a-registradora}

No trecho a seguir, um circuito quântico é inicializado com dois qubits
e sua representação gráfica é exibida no notebook por meio do comando
\texttt{qc.draw("mpl")}.

    \begin{tcolorbox}[breakable, size=fbox, boxrule=1pt, pad at break*=1mm,colback=cellbackground, colframe=cellborder]
\prompt{In}{incolor}{2}{\boxspacing}
\begin{Verbatim}[commandchars=\\\{\}]
\PY{c+c1}{\PYZsh{} Cria um circuito com 2 qubits}
\PY{n}{qc} \PY{o}{=} \PY{n}{QuantumCircuit}\PY{p}{(}\PY{l+m+mi}{2}\PY{p}{)}
\PY{c+c1}{\PYZsh{} Exibe o circuito}
\PY{n}{qc}\PY{o}{.}\PY{n}{draw}\PY{p}{(}\PY{l+s+s2}{\PYZdq{}}\PY{l+s+s2}{mpl}\PY{l+s+s2}{\PYZdq{}}\PY{p}{)}
\end{Verbatim}
\end{tcolorbox}
 
            
\prompt{Out}{outcolor}{2}{}
    
    \begin{center}
    \adjustimage{max size={0.9\linewidth}{0.9\paperheight}}{exemplo_circuito_01_files/exemplo_circuito_01_5_0.png}
    \end{center}
    { \hspace*{\fill} \\}
    

    No trecho a seguir, um circuito quântico é inicializado com dois qubits,
sua representação gráfica é exibida no notebook e o vetor de estado
associado às operações presentes no circuito é extraído. Esse vetor de
estado é então apresentado em notação LaTeX por meio de
\texttt{psi.draw(\textquotesingle{}latex\textquotesingle{})}. O vetor de
estado inicial é o \[|\psi\rangle = |00\rangle\]

    \begin{tcolorbox}[breakable, size=fbox, boxrule=1pt, pad at break*=1mm,colback=cellbackground, colframe=cellborder]
\prompt{In}{incolor}{3}{\boxspacing}
\begin{Verbatim}[commandchars=\\\{\}]
\PY{c+c1}{\PYZsh{} Extrai o vetor de estado resultante do circuito atual}
\PY{n}{psi} \PY{o}{=} \PY{n}{Statevector}\PY{o}{.}\PY{n}{from\PYZus{}instruction}\PY{p}{(}\PY{n}{qc}\PY{p}{)}
\PY{c+c1}{\PYZsh{}}
\PY{n}{psi}\PY{o}{.}\PY{n}{draw}\PY{p}{(}\PY{l+s+s1}{\PYZsq{}}\PY{l+s+s1}{latex}\PY{l+s+s1}{\PYZsq{}}\PY{p}{)}
\end{Verbatim}
\end{tcolorbox}
 
            
\prompt{Out}{outcolor}{3}{}
    
    $$ |00\rangle$$

    

    No trecho a seguir, o vetor de estado previamente obtido é representado
em sua forma textual em LaTeX. Essa representação é retornada por meio
do comando
\texttt{psi.draw(\textquotesingle{}latex\_source\textquotesingle{})}.

    \begin{tcolorbox}[breakable, size=fbox, boxrule=1pt, pad at break*=1mm,colback=cellbackground, colframe=cellborder]
\prompt{In}{incolor}{4}{\boxspacing}
\begin{Verbatim}[commandchars=\\\{\}]
\PY{c+c1}{\PYZsh{}}
\PY{n}{psi}\PY{o}{.}\PY{n}{draw}\PY{p}{(}\PY{l+s+s1}{\PYZsq{}}\PY{l+s+s1}{latex\PYZus{}source}\PY{l+s+s1}{\PYZsq{}}\PY{p}{)}
\end{Verbatim}
\end{tcolorbox}

            \begin{tcolorbox}[breakable, size=fbox, boxrule=.5pt, pad at break*=1mm, opacityfill=0]
\prompt{Out}{outcolor}{4}{\boxspacing}
\begin{Verbatim}[commandchars=\\\{\}]
' |00\textbackslash{}\textbackslash{}rangle'
\end{Verbatim}
\end{tcolorbox}
        
    No trecho a seguir, a esfera de Bloch correspondente ao vetor de estado
extraído do circuito é gerada e exibida no notebook. A visualização é
produzida pelo comando \texttt{plot\_bloch\_multivector(psi)} e
apresentada por meio de \texttt{display()}.

    \begin{tcolorbox}[breakable, size=fbox, boxrule=1pt, pad at break*=1mm,colback=cellbackground, colframe=cellborder]
\prompt{In}{incolor}{5}{\boxspacing}
\begin{Verbatim}[commandchars=\\\{\}]
\PY{c+c1}{\PYZsh{} plta esfrea de Bloch do estado do circuito}
\PY{n}{display}\PY{p}{(}\PY{n}{plot\PYZus{}bloch\PYZus{}multivector}\PY{p}{(}\PY{n}{psi}\PY{p}{)}\PY{p}{)}
\end{Verbatim}
\end{tcolorbox}

    \begin{center}
    \adjustimage{max size={0.9\linewidth}{0.9\paperheight}}{exemplo_circuito_01_files/exemplo_circuito_01_11_0.png}
    \end{center}
    { \hspace*{\fill} \\}
    
    \subsection{Aplicando a porta X no qubit
0}\label{aplicando-a-porta-x-no-qubit-0}

No trecho a seguir, um circuito quântico é inicializado com dois qubits,
a porta \(X\) é aplicada ao qubit \(0\) e a representação gráfica
resultante é exibida no notebook por meio de \texttt{qc.draw("mpl")}. Na
notação adotada pelo Qiskit, o estado computacional é escrito no formato
\(\lvert q_1 q_0\rangle\), de modo que o qubit \(0\) ocupa a posição
menos significativa e aparece na última posição do ket.

    \begin{tcolorbox}[breakable, size=fbox, boxrule=1pt, pad at break*=1mm,colback=cellbackground, colframe=cellborder]
\prompt{In}{incolor}{6}{\boxspacing}
\begin{Verbatim}[commandchars=\\\{\}]
\PY{c+c1}{\PYZsh{} Cria um circuito com 2 qubits}
\PY{n}{qc} \PY{o}{=} \PY{n}{QuantumCircuit}\PY{p}{(}\PY{l+m+mi}{2}\PY{p}{)}
\PY{c+c1}{\PYZsh{} Aplica a porta X no qubit 0}
\PY{n}{qc}\PY{o}{.}\PY{n}{x}\PY{p}{(}\PY{l+m+mi}{0}\PY{p}{)}
\PY{c+c1}{\PYZsh{} Exibe o circuito}
\PY{n}{qc}\PY{o}{.}\PY{n}{draw}\PY{p}{(}\PY{l+s+s2}{\PYZdq{}}\PY{l+s+s2}{mpl}\PY{l+s+s2}{\PYZdq{}}\PY{p}{)}
\end{Verbatim}
\end{tcolorbox}
 
            
\prompt{Out}{outcolor}{6}{}
    
    \begin{center}
    \adjustimage{max size={0.9\linewidth}{0.9\paperheight}}{exemplo_circuito_01_files/exemplo_circuito_01_13_0.png}
    \end{center}
    { \hspace*{\fill} \\}
    

    \subsection{Explicação teórica da aplicação de X no qubit
0}\label{explicauxe7uxe3o-teuxf3rica-da-aplicauxe7uxe3o-de-x-no-qubit-0}

No circuito considerado, trabalha-se no espaço de Hilbert
\(\mathcal{H} = \mathbb{C}^2 \otimes \mathbb{C}^2\), com base
computacional ordenada como

\[
\{|00\rangle, |01\rangle, |10\rangle, |11\rangle\},
\]

A base computacional de dois qubits é representada matricialmente por
vetores coluna de dimensão \(4\). Na convenção usual (ordenando os
estados como \(|00\rangle, |01\rangle, |10\rangle, |11\rangle\)), tem-se

\[
|00\rangle =
\begin{pmatrix}
1 \\ 0 \\ 0 \\ 0
\end{pmatrix},
\qquad
|01\rangle =
\begin{pmatrix}
0 \\ 1 \\ 0 \\ 0
\end{pmatrix},
\qquad
|10\rangle =
\begin{pmatrix}
0 \\ 0 \\ 1 \\ 0
\end{pmatrix},
\qquad
|11\rangle =
\begin{pmatrix}
0 \\ 0 \\ 0 \\ 1
\end{pmatrix}.
\]

e, na convenção do Qiskit, o ket é escrito como
\(\lvert q_1 q_0\rangle\), em que \(q_0\) ocupa a posição menos
significativa (última posição). A porta de Pauli-\(X\) é representada
pela matriz

\[
X =
\begin{pmatrix}
0 & 1 \\
1 & 0
\end{pmatrix},
\]

definida pelos autovetores da base computacional por

\[
X\lvert 0\rangle = \lvert 1\rangle,
\]

\[
X\lvert 1\rangle = \lvert 0\rangle.
\]

No circuito, a operação \(X\) é aplicada apenas ao qubit \(0\), logo o
operador global \(U\) (controlador padrão de portas) atuando em
\(\mathcal{H}\) é dado por

\[
U = I \otimes X,
\]

onde \(I\) é a matriz identidade em \(\mathbb{C}^2\) associada ao qubit
\(1\), e \(X\) é a porta aplicada no qubit \(0\). A forma matricial do
operador \(U = I \otimes X\) é obtida a partir das matrizes

\[
I =
\begin{pmatrix}
1 & 0 \\
0 & 1
\end{pmatrix},
\]

\[
X =
\begin{pmatrix}
0 & 1 \\
1 & 0
\end{pmatrix}.
\]

O produto de Kronecker resulta em

\[
U = I \otimes X =
\begin{pmatrix}
1\cdot X & 0\cdot X \\
0\cdot X & 1\cdot X
\end{pmatrix} = 
\begin{pmatrix}
X & 0 \\
0 & X
\end{pmatrix}.
\]

Substituindo explicitamente os blocos das submatrizes \(X\), obtém-se \[
U =
\begin{pmatrix}
0 & 1 & 0 & 0 \\
1 & 0 & 0 & 0 \\
0 & 0 & 0 & 1 \\
0 & 0 & 1 & 0
\end{pmatrix}.
\]

O estado inicial do registrador é dado por

\[
\lvert \psi_0\rangle = \lvert 00\rangle = \lvert 0\rangle_1 \otimes \lvert 0\rangle_0.
\]

A evolução unitária induzida por \(U\) é escrita como

\[
\begin{aligned}
\lvert \psi_1\rangle 
&= U \lvert \psi_0\rangle \\
&= (I \otimes X)\bigl(\lvert 0\rangle_1 \otimes \lvert 0\rangle_0\bigr) \\
&= \lvert 0\rangle_1 \otimes X\lvert 0\rangle_0 \\
&= \lvert 0\rangle_1 \otimes \lvert 1\rangle_0 \\
&= \lvert 01\rangle.
\end{aligned}
\]

Em representação de matrizes

\[
\lvert \psi_1\rangle
= U \lvert \psi_0\rangle
=
\begin{pmatrix}
0 & 1 & 0 & 0 \\
1 & 0 & 0 & 0 \\
0 & 0 & 0 & 1 \\
0 & 0 & 1 & 0
\end{pmatrix}
\begin{pmatrix}
1 \\ 0 \\ 0 \\ 0
\end{pmatrix}
=
\begin{pmatrix}
0 \\ 1 \\ 0 \\ 0
\end{pmatrix}
=
\lvert 01\rangle.
\]

O comando \texttt{Statevector.from\_instruction(qc)} constrói o vetor de
estado \(\lvert \psi_1\rangle\) associado ao circuito, e a instrução
\texttt{psi.draw(\textquotesingle{}latex\textquotesingle{})} exibe
exatamente o resultado \(\lvert 01\rangle\) na notação de kets adotada
pelo Qiskit.

    \begin{tcolorbox}[breakable, size=fbox, boxrule=1pt, pad at break*=1mm,colback=cellbackground, colframe=cellborder]
\prompt{In}{incolor}{7}{\boxspacing}
\begin{Verbatim}[commandchars=\\\{\}]
\PY{c+c1}{\PYZsh{} Extrai o vetor de estado resultante do circuito atual}
\PY{n}{psi} \PY{o}{=} \PY{n}{Statevector}\PY{o}{.}\PY{n}{from\PYZus{}instruction}\PY{p}{(}\PY{n}{qc}\PY{p}{)}
\PY{c+c1}{\PYZsh{}}
\PY{n}{psi}\PY{o}{.}\PY{n}{draw}\PY{p}{(}\PY{l+s+s1}{\PYZsq{}}\PY{l+s+s1}{latex}\PY{l+s+s1}{\PYZsq{}}\PY{p}{)}
\end{Verbatim}
\end{tcolorbox}
 
            
\prompt{Out}{outcolor}{7}{}
    
    $$ |01\rangle$$

    

    \subsection{A esfera de Bloch}\label{a-esfera-de-bloch}

A figura apresentada mostra a representação na esfera de Bloch dos dois
qubits que compõem o estado \(\lvert\psi_1\rangle = \lvert 01\rangle\).
Na convenção adotada, o qubit \(0\) corresponde ao último dígito do ket
e o qubit \(1\) ao primeiro. A esfera à esquerda representa o qubit
\(0\), cujo estado é \(\lvert 1\rangle\). Esse estado está associado ao
polo sul da esfera de Bloch, o que é indicado pela seta apontada para
baixo. A esfera à direita corresponde ao qubit \(1\), cujo estado é
\(\lvert 0\rangle\), localizado no polo norte da esfera de Bloch e
representado pela seta apontando para cima. Cada esfera exibe o estado
puro correspondente como um vetor unitário alinhado com o eixo \(z\),
refletindo que não há superposição entre \(\lvert 0\rangle\) e
\(\lvert 1\rangle\) em nenhum dos qubits.

    \begin{tcolorbox}[breakable, size=fbox, boxrule=1pt, pad at break*=1mm,colback=cellbackground, colframe=cellborder]
\prompt{In}{incolor}{8}{\boxspacing}
\begin{Verbatim}[commandchars=\\\{\}]
\PY{c+c1}{\PYZsh{} plta esfrea de Bloch do estado do circuito}
\PY{n}{display}\PY{p}{(}\PY{n}{plot\PYZus{}bloch\PYZus{}multivector}\PY{p}{(}\PY{n}{psi}\PY{p}{)}\PY{p}{)}
\end{Verbatim}
\end{tcolorbox}

    \begin{center}
    \adjustimage{max size={0.9\linewidth}{0.9\paperheight}}{exemplo_circuito_01_files/exemplo_circuito_01_17_0.png}
    \end{center}
    { \hspace*{\fill} \\}
    
    \subsection{Acrescentando Hadamard ao
circuito}\label{acrescentando-hadamard-ao-circuito}

O circuito exibido aplica duas operações sucessivas ao qubit \(0\). A
primeira operação é a porta \(X\), que transforma \(\lvert 0\rangle\) em
\(\lvert 1\rangle\) e vice-versa. Em seguida, aplica-se a porta de
Hadamard \(H\), que leva cada autoestado computacional para uma
superposição: \[
H\lvert 0\rangle = \frac{\lvert 0\rangle + \lvert 1\rangle}{\sqrt{2}},
\qquad
H\lvert 1\rangle = \frac{\lvert 0\rangle - \lvert 1\rangle}{\sqrt{2}}.
\] Como essas operações são aplicadas somente ao qubit \(0\), a linha
correspondente ao qubit \(1\) permanece inalterada. O diagrama produzido
reflete exatamente essa sequência: um bloco \(X\) seguido de um bloco
\(H\) na linha de \(q_0\), enquanto a linha de \(q_1\) permanece vazia,
indicando ausência de operações.

    \begin{tcolorbox}[breakable, size=fbox, boxrule=1pt, pad at break*=1mm,colback=cellbackground, colframe=cellborder]
\prompt{In}{incolor}{9}{\boxspacing}
\begin{Verbatim}[commandchars=\\\{\}]
\PY{c+c1}{\PYZsh{} Cria um circuito com 2 qubits}
\PY{n}{qc} \PY{o}{=} \PY{n}{QuantumCircuit}\PY{p}{(}\PY{l+m+mi}{2}\PY{p}{)}
\PY{c+c1}{\PYZsh{} Aplica a porta X no qubit 0}
\PY{n}{qc}\PY{o}{.}\PY{n}{x}\PY{p}{(}\PY{l+m+mi}{0}\PY{p}{)}
\PY{c+c1}{\PYZsh{} Aplica uma porta Hadamard (H) no qubit 0}
\PY{n}{qc}\PY{o}{.}\PY{n}{h}\PY{p}{(}\PY{l+m+mi}{0}\PY{p}{)}
\PY{c+c1}{\PYZsh{} Exibe o circuito}
\PY{n}{qc}\PY{o}{.}\PY{n}{draw}\PY{p}{(}\PY{l+s+s2}{\PYZdq{}}\PY{l+s+s2}{mpl}\PY{l+s+s2}{\PYZdq{}}\PY{p}{)}
\end{Verbatim}
\end{tcolorbox}
 
            
\prompt{Out}{outcolor}{9}{}
    
    \begin{center}
    \adjustimage{max size={0.9\linewidth}{0.9\paperheight}}{exemplo_circuito_01_files/exemplo_circuito_01_19_0.png}
    \end{center}
    { \hspace*{\fill} \\}
    

    \subsection{Explicação teórica da atuação de
Hadamard}\label{explicauxe7uxe3o-teuxf3rica-da-atuauxe7uxe3o-de-hadamard}

O vetor de estado obtido corresponde ao resultado da aplicação
sequencial das portas \(X\) e \(H\) no qubit \(0\), com o qubit \(1\)
permanecendo no estado \(\lvert 0\rangle\). O estado inicial do
registrador é

\[
\lvert \psi_0\rangle = \lvert 00\rangle.
\]

A porta \(X\) atua apenas no qubit menos significativo, produzindo

\[
(I \otimes X)\lvert 00\rangle = \lvert 01\rangle.
\]

Em seguida, a porta de Hadamard atua no mesmo qubit. Usando

\[
H\lvert 0\rangle = \frac{\lvert 0\rangle + \lvert 1\rangle}{\sqrt{2}},
\qquad
H\lvert 1\rangle = \frac{\lvert 0\rangle - \lvert 1\rangle}{\sqrt{2}},
\]

obtém-se

\[
\begin{aligned}
|\psi_2\rangle &= (I \otimes H)\lvert 01\rangle \\
&= \lvert 0\rangle_1 \otimes H\lvert 1\rangle_0 \\
&= \lvert 0\rangle_1 \otimes \frac{\lvert 0\rangle_0 + \lvert 1\rangle_0}{\sqrt{2}}\\
&= \frac{1}{\sqrt{2}}\bigl(\lvert 00\rangle - \lvert 01\rangle\bigr).
\end{aligned}
\]

Esse é exatamente o estado exibido pelo comando
\texttt{psi.draw(\textquotesingle{}latex\textquotesingle{})},
representado como

\[
|\psi_2\rangle = \frac{\sqrt{2}}{2}\lvert 00\rangle-
\frac{\sqrt{2}}{2}\lvert 01\rangle.
\]

    \begin{tcolorbox}[breakable, size=fbox, boxrule=1pt, pad at break*=1mm,colback=cellbackground, colframe=cellborder]
\prompt{In}{incolor}{10}{\boxspacing}
\begin{Verbatim}[commandchars=\\\{\}]
\PY{c+c1}{\PYZsh{} Extrai o vetor de estado resultante do circuito atual}
\PY{n}{psi} \PY{o}{=} \PY{n}{Statevector}\PY{o}{.}\PY{n}{from\PYZus{}instruction}\PY{p}{(}\PY{n}{qc}\PY{p}{)}
\PY{c+c1}{\PYZsh{}}
\PY{n}{psi}\PY{o}{.}\PY{n}{draw}\PY{p}{(}\PY{l+s+s1}{\PYZsq{}}\PY{l+s+s1}{latex}\PY{l+s+s1}{\PYZsq{}}\PY{p}{)}
\end{Verbatim}
\end{tcolorbox}
 
            
\prompt{Out}{outcolor}{10}{}
    
    $$\frac{\sqrt{2}}{2} |00\rangle- \frac{\sqrt{2}}{2} |01\rangle$$

    

    \subsection{Estado em superposição na esfera de
Bloch}\label{estado-em-superposiuxe7uxe3o-na-esfera-de-bloch}

A imagem mostra a representação de Bloch de cada qubit no estado

\[
\lvert \psi \rangle = \frac{1}{\sqrt{2}}\bigl(\lvert 00\rangle - \lvert 01\rangle\bigr).
\]

Nesse estado, o qubit \(1\) permanece em \(\lvert 0\rangle\), pois o
primeiro dígito do ket não é afetado pelas operações realizadas. Na
esfera à direita, isso aparece como um vetor alinhado com o polo norte,
indicando ausência de superposição desse qubit. Já o qubit \(0\)
encontra-se no estado

\[
H\lvert 1\rangle = \frac{1}{\sqrt{2}}\bigl(\lvert 0\rangle - \lvert 1\rangle\bigr),
\]

que corresponde a um estado puro com componentes reais e fase relativa
negativa entre \(\lvert 0\rangle\) e \(\lvert 1\rangle\). A esfera à
esquerda mostra esse estado como um vetor inclinado no semiplano \(xz\),
refletindo que o qubit está em superposição não trivial entre os
autoestados computacionais. Assim, cada esfera exibe a posição
geométrica do estado individual de cada qubit, obtida após decompor o
estado global em seus estados reduzidos.

    \begin{tcolorbox}[breakable, size=fbox, boxrule=1pt, pad at break*=1mm,colback=cellbackground, colframe=cellborder]
\prompt{In}{incolor}{11}{\boxspacing}
\begin{Verbatim}[commandchars=\\\{\}]
\PY{c+c1}{\PYZsh{} plta esfrea de Bloch do estado do circuito}
\PY{n}{display}\PY{p}{(}\PY{n}{plot\PYZus{}bloch\PYZus{}multivector}\PY{p}{(}\PY{n}{psi}\PY{p}{)}\PY{p}{)}
\end{Verbatim}
\end{tcolorbox}

    \begin{center}
    \adjustimage{max size={0.9\linewidth}{0.9\paperheight}}{exemplo_circuito_01_files/exemplo_circuito_01_23_0.png}
    \end{center}
    { \hspace*{\fill} \\}
    
    \subsection{Acrescentando a porta CNOT ao
circuito}\label{acrescentando-a-porta-cnot-ao-circuito}

O circuito apresentado aplica três operações sucessivas. Primeiro, a
porta \(X\) é aplicada ao qubit \(0\), invertendo seu estado. Em
seguida, a porta de Hadamard \(H\) atua no mesmo qubit, produzindo uma
superposição linear dos estados computacionais. Após essas operações
locais, é aplicada a porta \(CNOT\), na qual o qubit \(0\) funciona como
controle e o qubit \(1\) funciona como alvo. O que implica que o segundo
qubit é invertido se, e somente se, o primeiro qubit estiver no estado
\(\lvert 1\rangle\). Em termos descritivos, a porta \(CNOT\) opera em um
registrador quântico de dois qubits e realiza um flip no qubit alvo
exatamente quando o qubit de controle ocupa o estado
\(\lvert 1\rangle\).

\begin{longtable}[]{@{}
  >{\raggedright\arraybackslash}p{(\linewidth - 6\tabcolsep) * \real{0.2895}}
  >{\raggedright\arraybackslash}p{(\linewidth - 6\tabcolsep) * \real{0.2105}}
  >{\raggedright\arraybackslash}p{(\linewidth - 6\tabcolsep) * \real{0.2895}}
  >{\raggedright\arraybackslash}p{(\linewidth - 6\tabcolsep) * \real{0.2105}}@{}}
\toprule\noalign{}
\begin{minipage}[b]{\linewidth}\raggedright
\textbf{Antes}
\end{minipage} & \begin{minipage}[b]{\linewidth}\raggedright
\end{minipage} & \begin{minipage}[b]{\linewidth}\raggedright
\textbf{Depois}
\end{minipage} & \begin{minipage}[b]{\linewidth}\raggedright
\end{minipage} \\
\midrule\noalign{}
\endhead
\bottomrule\noalign{}
\endlastfoot
Controle & Alvo & Controle & Alvo \\
\(\lvert 0\rangle\) & \(\lvert 0\rangle\) & \(\lvert 0\rangle\) &
\(\lvert 0\rangle\) \\
\(\lvert 0\rangle\) & \(\lvert 1\rangle\) & \(\lvert 0\rangle\) &
\(\lvert 1\rangle\) \\
\(\lvert 1\rangle\) & \(\lvert 0\rangle\) & \(\lvert 1\rangle\) &
\(\lvert 1\rangle\) \\
\(\lvert 1\rangle\) & \(\lvert 1\rangle\) & \(\lvert 1\rangle\) &
\(\lvert 0\rangle\) \\
\end{longtable}

Se \(\{\lvert 0\rangle , \lvert 1\rangle\}\) são os únicos valores de
entrada permitidos para ambos os qubits, então a saída do \textbf{qubit
alvo} da porta CNOT corresponde ao resultado de uma porta XOR clássica.
Mantendo o \textbf{qubit de controle} fixado em \(\lvert 1\rangle\), a
saída do qubit alvo da porta CNOT produz o resultado de uma porta NOT
clássica. De forma mais geral, as entradas podem estar em uma
superposição linear dos estados \({\lvert 0\rangle , \lvert 1\rangle}\).
A porta CNOT transforma o estado quântico

\[
a\lvert 00\rangle + b\lvert 01\rangle + c\lvert 10\rangle + d\lvert 11\rangle
\]

em

\[
a\lvert 00\rangle + b\lvert 01\rangle + c\lvert 11\rangle + d\lvert 10\rangle.
\]

A ação da porta CNOT pode ser representada pela matriz (na forma de uma
matriz de permutação):

\[
\mathrm{CNOT} =
\begin{bmatrix}
1 & 0 & 0 & 0 \\
0 & 1 & 0 & 0 \\
0 & 0 & 0 & 1 \\
0 & 0 & 1 & 0
\end{bmatrix}.
\]

O diagrama exibido no notebook reflete essa estrutura: um bloco \(X\)
seguido de um bloco \(H\) na linha de \(q_0\), e a operação controlada
conectando \(q_0\) ao qubit \(1\) por meio do símbolo de controle na
linha superior e do símbolo de soma na linha inferior.

    \begin{tcolorbox}[breakable, size=fbox, boxrule=1pt, pad at break*=1mm,colback=cellbackground, colframe=cellborder]
\prompt{In}{incolor}{12}{\boxspacing}
\begin{Verbatim}[commandchars=\\\{\}]
\PY{c+c1}{\PYZsh{} Cria um circuito com 2 qubits}
\PY{n}{qc} \PY{o}{=} \PY{n}{QuantumCircuit}\PY{p}{(}\PY{l+m+mi}{2}\PY{p}{)}
\PY{c+c1}{\PYZsh{} Aplica a porta X no qubit 0}
\PY{n}{qc}\PY{o}{.}\PY{n}{x}\PY{p}{(}\PY{l+m+mi}{0}\PY{p}{)}
\PY{c+c1}{\PYZsh{} Aplica uma porta Hadamard (H) no qubit 0}
\PY{n}{qc}\PY{o}{.}\PY{n}{h}\PY{p}{(}\PY{l+m+mi}{0}\PY{p}{)}
\PY{c+c1}{\PYZsh{} Aplica uma porta CNOT com controle em q0 e alvo em q1}
\PY{n}{qc}\PY{o}{.}\PY{n}{cx}\PY{p}{(}\PY{l+m+mi}{0}\PY{p}{,} \PY{l+m+mi}{1}\PY{p}{)}
\PY{c+c1}{\PYZsh{} Exibe o circuito}
\PY{n}{qc}\PY{o}{.}\PY{n}{draw}\PY{p}{(}\PY{l+s+s2}{\PYZdq{}}\PY{l+s+s2}{mpl}\PY{l+s+s2}{\PYZdq{}}\PY{p}{)}
\end{Verbatim}
\end{tcolorbox}
 
            
\prompt{Out}{outcolor}{12}{}
    
    \begin{center}
    \adjustimage{max size={0.9\linewidth}{0.9\paperheight}}{exemplo_circuito_01_files/exemplo_circuito_01_25_0.png}
    \end{center}
    { \hspace*{\fill} \\}
    

    \subsection{Estado entrelaçado}\label{estado-entrelauxe7ado}

Esse resultado corresponde ao estado produzido pela sequência de
operações aplicadas no circuito: a porta \(X\), seguida da porta de
Hadamard \(H\) no qubit \(0\), e por fim a porta CNOT com controle em
\(q_0\) e alvo em \(q_1\). O estado inicial do registrador é

\[
\lvert\psi_0\rangle = \lvert 00\rangle.
\]

Após a aplicação de \(X\) no qubit menos significativo, obtém-se

\[
(I\otimes X)\lvert 00\rangle = \lvert 01\rangle.
\]

Em seguida, a porta Hadamard atua nesse mesmo qubit:

\[
H\lvert 1\rangle = \frac{\lvert 0\rangle - \lvert 1\rangle}{\sqrt{2}},
\]

o que produz

\[
(I\otimes H)\lvert 01\rangle
= \frac{1}{\sqrt{2}}\bigl(\lvert 00\rangle - \lvert 01\rangle\bigr).
\]

Por fim, a porta CNOT atua no circuito, e os termos individuais são
transformados como

\[
\lvert 00\rangle \mapsto \lvert 00\rangle,\qquad
\lvert 01\rangle \mapsto \lvert 11\rangle,
\]

pois apenas no segundo termo o qubit de controle está em
\(\lvert 1\rangle\). Assim, o estado final é

\[
\lvert\psi\rangle
= \frac{1}{\sqrt{2}}\lvert 00\rangle -
\frac{1}{\sqrt{2}}\lvert 11\rangle,
\]

que é exatamente o resultado exibido pelo comando
\texttt{psi.draw(\textquotesingle{}latex\textquotesingle{})}.

    \begin{tcolorbox}[breakable, size=fbox, boxrule=1pt, pad at break*=1mm,colback=cellbackground, colframe=cellborder]
\prompt{In}{incolor}{13}{\boxspacing}
\begin{Verbatim}[commandchars=\\\{\}]
\PY{c+c1}{\PYZsh{} Extrai o vetor de estado resultante do circuito atual}
\PY{n}{psi} \PY{o}{=} \PY{n}{Statevector}\PY{o}{.}\PY{n}{from\PYZus{}instruction}\PY{p}{(}\PY{n}{qc}\PY{p}{)}
\PY{c+c1}{\PYZsh{}}
\PY{n}{psi}\PY{o}{.}\PY{n}{draw}\PY{p}{(}\PY{l+s+s1}{\PYZsq{}}\PY{l+s+s1}{latex}\PY{l+s+s1}{\PYZsq{}}\PY{p}{)}
\end{Verbatim}
\end{tcolorbox}
 
            
\prompt{Out}{outcolor}{13}{}
    
    $$\frac{\sqrt{2}}{2} |00\rangle- \frac{\sqrt{2}}{2} |11\rangle$$

    

    \subsection{Estado entrelaçado não pode ser representado na esfera de
Bloch}\label{estado-entrelauxe7ado-nuxe3o-pode-ser-representado-na-esfera-de-bloch}

Esse estado final,

\[
\lvert\psi\rangle
= \frac{1}{\sqrt{2}}\lvert 00\rangle -
\frac{1}{\sqrt{2}}\lvert 11\rangle,
\]

corresponde a um estado emaranhado de dois qubits. Cada termo da
superposição envolve ambos os qubits simultaneamente, e por isso o
estado não pode ser escrito como um produto de estados individuais, isto
é,

\[
\lvert\psi\rangle \neq \lvert\phi_1\rangle \otimes \lvert\phi_0\rangle.
\]

A representação na esfera de Bloch exige exatamente essa fatorabilidade:
a esfera de Bloch descreve apenas \textbf{estados puros de um único
qubit}, que pertencem ao espaço de Hilbert de dimensão \(2\). No caso
presente, o estado pertence ao espaço \(\mathbb{C}^4\) e não pode ser
reduzido a estados puros de cada qubit separadamente. De fato, ao tentar
obter os estados individuais, deve-se calcular as matrizes densidade
reduzidas por meio da operação traço parcial. Para um estado emaranhado
como este, cada qubit isolado encontra-se em um estado \textbf{misto},
descrito por uma matriz densidade que não é um projetor de posto \(1\).
Por exemplo, para este estado específico tem-se que

\[
\rho_0 = \operatorname{Tr}*{q_1}(\lvert\psi\rangle\langle\psi\rvert)
= \frac{1}{2} I,
\]

\[
\rho_1 = \operatorname{Tr}*{q_0}(\lvert\psi\rangle\langle\psi\rvert)
= \frac{1}{2} I,
\]

sendo \(I\) a matriz identidade \(2\times 2\). Esses estados são
puramente mistos e correspondem ao \textbf{centro} da esfera de Bloch,
não a um vetor localizado na superfície. Portanto, não é possível
representar o estado global \(\lvert\psi\rangle\) em uma esfera de
Bloch, pois a esfera descreve apenas estados puros de um único qubit. A
esfera só pode ser usada para representar os estados reduzidos de cada
qubit, que neste caso são mistos e aparecem como o ponto central da
esfera, sem direção definida.

    \begin{tcolorbox}[breakable, size=fbox, boxrule=1pt, pad at break*=1mm,colback=cellbackground, colframe=cellborder]
\prompt{In}{incolor}{14}{\boxspacing}
\begin{Verbatim}[commandchars=\\\{\}]
\PY{c+c1}{\PYZsh{} plta esfrea de Bloch do estado do circuito}
\PY{n}{display}\PY{p}{(}\PY{n}{plot\PYZus{}bloch\PYZus{}multivector}\PY{p}{(}\PY{n}{psi}\PY{p}{)}\PY{p}{)}
\end{Verbatim}
\end{tcolorbox}

    \begin{center}
    \adjustimage{max size={0.9\linewidth}{0.9\paperheight}}{exemplo_circuito_01_files/exemplo_circuito_01_29_0.png}
    \end{center}
    { \hspace*{\fill} \\}
    
    \subsection{Acrescentando rotação em y ao
circuito}\label{acrescentando-rotauxe7uxe3o-em-y-ao-circuito}

O circuito mostrado aplica uma sequência de quatro operações. Primeiro,
o qubit \(0\) recebe a porta \(X\), que inverte seu estado. Em seguida,
aplica-se a porta de Hadamard \(H\) no mesmo qubit, colocando-o em
superposição. Depois dessas operações locais, a porta CNOT é aplicada,
com \(q_0\) atuando como qubit de controle e \(q_1\) como qubit alvo;
nessa operação, o alvo é invertido somente quando o controle está no
estado \(\lvert 1\rangle\). Por fim, o qubit \(1\) recebe a rotação
\(R_y(\pi)\), que corresponde a uma rotação de ângulo \(\pi\) em torno
do eixo \(y\). A operação \(R_y(\theta)\) é definida a partir das
matrizes de Pauli. Em particular, a matriz de Pauli associada ao eixo
\(y\) é

\[
\sigma_y =
\begin{pmatrix}
0 & -i \\
i & 0
\end{pmatrix},
\]

uma das três matrizes de Pauli \({\sigma_x, \sigma_y, \sigma_z}\), que
formam uma base para o espaço dos operadores hermitianos \(2\times 2\) e
representam rotações fundamentais em \(\mathbb{C}^2\). Essas matrizes
são geradoras das rotações em \(SU(2)\), e o operador de rotação em
torno do eixo \(y\) é definido como

\[
R_y(\theta) = e^{-i\theta \sigma_y/2}.
\]

A exponenciação de operadores leva à forma fechada

\[
R_y(\theta) =
\begin{pmatrix}
\cos(\theta/2) & -\sin(\theta/2) \\
\sin(\theta/2) & \cos(\theta/2)
\end{pmatrix},
\]

que, para \(\theta = \pi\), resulta em

\[
R_y(\pi) =
\begin{pmatrix}
0 & -1 \\
1 & 0
\end{pmatrix}.
\]

Essa operação atua nos estados computacionais como

\[
R_y(\pi)\lvert 0\rangle = \lvert 1\rangle,
\]

\[
R_y(\pi)\lvert 1\rangle = -\lvert 0\rangle,
\]

mostrando que a rotação de \(\pi\) em torno de \(y\) produz
essencialmente uma inversão dos estados, com uma fase relativa no
segundo caso. O diagrama do circuito representa exatamente essa
sequência: as portas \(X\) e \(H\) em \(q_0\), a porta CNOT conectando
os dois qubits e, ao final, a rotação \(R_y(\pi)\) aplicada ao qubit
\(1\).

    \begin{tcolorbox}[breakable, size=fbox, boxrule=1pt, pad at break*=1mm,colback=cellbackground, colframe=cellborder]
\prompt{In}{incolor}{15}{\boxspacing}
\begin{Verbatim}[commandchars=\\\{\}]
\PY{c+c1}{\PYZsh{} Cria um circuito com 2 qubits}
\PY{n}{qc} \PY{o}{=} \PY{n}{QuantumCircuit}\PY{p}{(}\PY{l+m+mi}{2}\PY{p}{)}
\PY{c+c1}{\PYZsh{} Aplica a porta X no qubit 0}
\PY{n}{qc}\PY{o}{.}\PY{n}{x}\PY{p}{(}\PY{l+m+mi}{0}\PY{p}{)}
\PY{c+c1}{\PYZsh{} Aplica uma porta Hadamard (H) no qubit 0}
\PY{n}{qc}\PY{o}{.}\PY{n}{h}\PY{p}{(}\PY{l+m+mi}{0}\PY{p}{)}
\PY{c+c1}{\PYZsh{} Aplica uma porta CNOT com controle em q0 e alvo em q1}
\PY{n}{qc}\PY{o}{.}\PY{n}{cx}\PY{p}{(}\PY{l+m+mi}{0}\PY{p}{,} \PY{l+m+mi}{1}\PY{p}{)}
\PY{c+c1}{\PYZsh{} Aplica uma rotação R\PYZus{}y(π) no qubit 1}
\PY{n}{qc}\PY{o}{.}\PY{n}{ry}\PY{p}{(}\PY{n}{np}\PY{o}{.}\PY{n}{pi}\PY{p}{,} \PY{l+m+mi}{1}\PY{p}{)}
\PY{c+c1}{\PYZsh{} Exibe o circuito}
\PY{n}{qc}\PY{o}{.}\PY{n}{draw}\PY{p}{(}\PY{l+s+s2}{\PYZdq{}}\PY{l+s+s2}{mpl}\PY{l+s+s2}{\PYZdq{}}\PY{p}{)}
\end{Verbatim}
\end{tcolorbox}
 
            
\prompt{Out}{outcolor}{15}{}
    
    \begin{center}
    \adjustimage{max size={0.9\linewidth}{0.9\paperheight}}{exemplo_circuito_01_files/exemplo_circuito_01_31_0.png}
    \end{center}
    { \hspace*{\fill} \\}
    

    \subsection{Explicação teórica sobre a atuação da rotação em y no
circuito}\label{explicauxe7uxe3o-teuxf3rica-sobre-a-atuauxe7uxe3o-da-rotauxe7uxe3o-em-y-no-circuito}

Esse vetor de estado corresponde ao resultado da aplicação sequencial
das quatro operações do circuito: as portas \(X\) e \(H\) no qubit
\(0\), o CNOT com controle em \(q_0\) e alvo em \(q_1\), e finalmente a
rotação \(R_y(\pi)\) no qubit \(1\). O estado inicial é

\[
\lvert\psi_0\rangle = \lvert 00\rangle.
\]

Após a porta \(X\) em \(q_0\), obtém-se

\[
\lvert\psi_1\rangle =\lvert 01\rangle.
\]

A porta Hadamard no qubit \(0\) produz

\[
\lvert\psi_2\rangle =
\frac{1}{\sqrt{2}}\lvert 00\rangle -
\frac{1}{\sqrt{2}}\lvert 01\rangle.
\]

A seguir, a porta CNOT resulta em

\[
\lvert\psi_3\rangle =
\frac{1}{\sqrt{2}}\lvert 00\rangle-
\frac{1}{\sqrt{2}}\lvert 11\rangle.
\]

Por fim, aplica-se a rotação \(R_y(\pi)\) no qubit \(1\) resulta em

\[
\begin{aligned}
\lvert\psi_4\rangle &= R_y(\pi) \lvert\psi_3\rangle \\
&= R_y(\pi) \left[ \frac{1}{\sqrt{2}}\lvert 0\rangle_1 \otimes |0 \rangle_0 - \frac{1}{\sqrt{2}}\lvert 1\rangle_1 \otimes |1\rangle_0\right]. \\
&= \frac{1}{\sqrt{2}}\lvert 1\rangle_1 \otimes |0 \rangle_0 + \frac{1}{\sqrt{2}}\lvert 0\rangle_1 \otimes |1\rangle_0 \\
&= \frac{1}{\sqrt{2}}\lvert 01\rangle + \frac{1}{\sqrt{2}}\lvert 10\rangle.
\end{aligned}
\]

o que é matematicamente equivalente à forma simplificada acima. Esse é
exatamente o estado exibido pelo comando
\texttt{psi.draw(\textquotesingle{}latex\textquotesingle{})}.

    \begin{tcolorbox}[breakable, size=fbox, boxrule=1pt, pad at break*=1mm,colback=cellbackground, colframe=cellborder]
\prompt{In}{incolor}{16}{\boxspacing}
\begin{Verbatim}[commandchars=\\\{\}]
\PY{c+c1}{\PYZsh{} Extrai o vetor de estado resultante do circuito atual}
\PY{n}{psi} \PY{o}{=} \PY{n}{Statevector}\PY{o}{.}\PY{n}{from\PYZus{}instruction}\PY{p}{(}\PY{n}{qc}\PY{p}{)}
\PY{c+c1}{\PYZsh{}}
\PY{n}{psi}\PY{o}{.}\PY{n}{draw}\PY{p}{(}\PY{l+s+s1}{\PYZsq{}}\PY{l+s+s1}{latex}\PY{l+s+s1}{\PYZsq{}}\PY{p}{)}
\end{Verbatim}
\end{tcolorbox}
 
            
\prompt{Out}{outcolor}{16}{}
    
    $$\frac{\sqrt{2}}{2} |01\rangle+\frac{\sqrt{2}}{2} |10\rangle$$

    

    \subsection{Acrescentando mais uma porta
Hadamard}\label{acrescentando-mais-uma-porta-hadamard}

O circuito aplica cinco operações sucessivas sobre dois qubits.
Primeiro, o qubit \(q_0\) recebe a porta \(X\), que inverte seu estado.
Em seguida, aplica-se uma porta de Hadamard \(H\) no mesmo qubit,
colocando-o em superposição. Depois, o circuito aplica uma porta CNOT,
na qual \(q_0\) atua como controle e \(q_1\) atua como alvo; o alvo é
invertido somente quando o controle está no estado \(\lvert 1\rangle\).
A operação seguinte é uma rotação \(R_y(\pi)\) aplicada ao qubit
\(q_1\), implementando uma rotação de ângulo \(\pi\) em torno do eixo
\(y\), definida pelo operador

\[
R_y(\pi)=
\exp\left(-\frac{i\pi}{2}\sigma_y\right) =
\begin{pmatrix}
0 & -1 \\
1 & 0
\end{pmatrix},
\]

sendo \(\sigma_y\) a matriz de Pauli correspondente ao eixo \(y\),

\[
\sigma_y =
\begin{pmatrix}
0 & -i \\
i & 0
\end{pmatrix}.
\]

Por fim, uma segunda porta Hadamard é aplicada novamente ao qubit
\(q_0\). O diagrama gerado pelo Qiskit reflete exatamente essa
sequência, com a linha superior mostrando as operações locais e o ponto
de controle do CNOT, e a linha inferior mostrando o alvo do CNOT e a
rotação \(R_y(\pi)\).

    \begin{tcolorbox}[breakable, size=fbox, boxrule=1pt, pad at break*=1mm,colback=cellbackground, colframe=cellborder]
\prompt{In}{incolor}{17}{\boxspacing}
\begin{Verbatim}[commandchars=\\\{\}]
\PY{c+c1}{\PYZsh{} Cria um circuito com 2 qubits}
\PY{n}{qc} \PY{o}{=} \PY{n}{QuantumCircuit}\PY{p}{(}\PY{l+m+mi}{2}\PY{p}{)}
\PY{c+c1}{\PYZsh{} Aplica a porta X no qubit 0}
\PY{n}{qc}\PY{o}{.}\PY{n}{x}\PY{p}{(}\PY{l+m+mi}{0}\PY{p}{)}
\PY{c+c1}{\PYZsh{} Aplica uma porta Hadamard (H) no qubit 0}
\PY{n}{qc}\PY{o}{.}\PY{n}{h}\PY{p}{(}\PY{l+m+mi}{0}\PY{p}{)}
\PY{c+c1}{\PYZsh{} Aplica uma porta CNOT com controle em q0 e alvo em q1}
\PY{n}{qc}\PY{o}{.}\PY{n}{cx}\PY{p}{(}\PY{l+m+mi}{0}\PY{p}{,} \PY{l+m+mi}{1}\PY{p}{)}
\PY{c+c1}{\PYZsh{} Aplica uma rotação R\PYZus{}y(π) no qubit 1}
\PY{n}{qc}\PY{o}{.}\PY{n}{ry}\PY{p}{(}\PY{n}{np}\PY{o}{.}\PY{n}{pi}\PY{p}{,} \PY{l+m+mi}{1}\PY{p}{)}
\PY{c+c1}{\PYZsh{} Aplica novamente uma porta Hadamard no qubit 0}
\PY{n}{qc}\PY{o}{.}\PY{n}{h}\PY{p}{(}\PY{l+m+mi}{0}\PY{p}{)}
\PY{c+c1}{\PYZsh{} Exibe o circuito}
\PY{n}{qc}\PY{o}{.}\PY{n}{draw}\PY{p}{(}\PY{l+s+s2}{\PYZdq{}}\PY{l+s+s2}{mpl}\PY{l+s+s2}{\PYZdq{}}\PY{p}{)}
\end{Verbatim}
\end{tcolorbox}
 
            
\prompt{Out}{outcolor}{17}{}
    
    \begin{center}
    \adjustimage{max size={0.9\linewidth}{0.9\paperheight}}{exemplo_circuito_01_files/exemplo_circuito_01_35_0.png}
    \end{center}
    { \hspace*{\fill} \\}
    

    O vetor de estado apresentado pelo Qiskit corresponde ao resultado da
aplicação sucessiva das cinco operações do circuito. O registrador é
inicializado no estado \(\lvert 00\rangle\), e a porta \(X\) aplicada ao
qubit \(q_0\) produz o estado \(\lvert 01\rangle\). Em seguida, a porta
de Hadamard é aplicada ao mesmo qubit, fazendo com que o registrador
passe a ser descrito por

\[
\frac{1}{\sqrt{2}}\bigl(\lvert 00\rangle - \lvert 01\rangle\bigr).
\]

A porta CNOT é então aplicada, com \(q_0\) atuando como qubit de
controle e \(q_1\) como alvo, de modo que apenas o segundo termo da
superposição é modificado. Como o controle encontra-se em
\(\lvert 1\rangle\) nesse termo, o alvo é invertido, e o estado passa a
ser

\[
\frac{1}{\sqrt{2}}\bigl(\lvert 00\rangle - \lvert 11\rangle\bigr).
\]

A operação seguinte consiste na rotação \(R_y(\pi)\) aplicada ao qubit
\(q_1\). Essa rotação é definida pela ação \$\$

\[
R\_y(\pi)\lvert 0\rangle = \lvert 1\rangle, \qquad
R\_y(\pi)\lvert 1\rangle = -\lvert 0\rangle, \$\$
\]

de modo que os termos da superposição são transformados segundo

\[
\lvert 00\rangle \mapsto \lvert 10\rangle,
\qquad
\lvert 11\rangle \mapsto -\lvert 01\rangle.
\]

Como consequência, o registrador passa a ser descrito por

\[
\frac{1}{\sqrt{2}}\bigl(\lvert 10\rangle + \lvert 01\rangle\bigr).
\]

Finalmente, uma segunda porta de Hadamard é aplicada ao qubit \(q_0\).
Utilizando-se as transformações

\[
H\lvert 0\rangle = \frac{\lvert 0\rangle + \lvert 1\rangle}{\sqrt{2}},
\qquad
H\lvert 1\rangle = \frac{\lvert 0\rangle - \lvert 1\rangle}{\sqrt{2}},
\]

a superposição anterior é convertida no estado

\[
\frac{1}{2}\bigl(\lvert 00\rangle - \lvert 01\rangle + \lvert 10\rangle + \lvert 11\rangle\bigr),
\]

o qual coincide exatamente com o vetor exibido pelo comando
\texttt{psi.draw(\textquotesingle{}latex\textquotesingle{})}.

    \begin{tcolorbox}[breakable, size=fbox, boxrule=1pt, pad at break*=1mm,colback=cellbackground, colframe=cellborder]
\prompt{In}{incolor}{18}{\boxspacing}
\begin{Verbatim}[commandchars=\\\{\}]
\PY{c+c1}{\PYZsh{} Extrai o vetor de estado resultante do circuito atual}
\PY{n}{psi} \PY{o}{=} \PY{n}{Statevector}\PY{o}{.}\PY{n}{from\PYZus{}instruction}\PY{p}{(}\PY{n}{qc}\PY{p}{)}
\PY{c+c1}{\PYZsh{}}
\PY{n}{psi}\PY{o}{.}\PY{n}{draw}\PY{p}{(}\PY{l+s+s1}{\PYZsq{}}\PY{l+s+s1}{latex}\PY{l+s+s1}{\PYZsq{}}\PY{p}{)}
\end{Verbatim}
\end{tcolorbox}
 
            
\prompt{Out}{outcolor}{18}{}
    
    $$\frac{1}{2} |00\rangle- \frac{1}{2} |01\rangle+\frac{1}{2} |10\rangle+\frac{1}{2} |11\rangle$$

    

    \section{Apêndice: traço parcial e estados
mistos}\label{apuxeandice-trauxe7o-parcial-e-estados-mistos}

Tomar o \textbf{traço parcial} de um estado composto significa descartar
matematicamente uma das partes do sistema quântico para obter a
descrição do que resta. Quando dois qubits formam um estado conjunto
\(\rho\), eles vivem no espaço de Hilbert
\(\mathcal{H} = \mathcal{H}_1 \otimes \mathcal{H}_0\). No entanto, se
estamos interessados apenas em um dos qubits --- por exemplo, o qubit
\(0\) --- precisamos eliminar a informação do qubit \(1\). Esse processo
é descrito pela operação chamada \textbf{traço parcial}, denotada por

\[
\rho_0 = \operatorname{Tr}_{q_1}(\rho).
\]

O traço parcial combina duas ideias: (1) não estamos observando o qubit
descartado; (2) toda informação que depende exclusivamente desse qubit é
eliminada. Em termos formais, ele atua somando todas as componentes de
\(\rho\) que pertencem ao qubit descartado, preservando apenas a
estrutura matemática relevante para o qubit restante. No caso de um
estado puro de dois qubits,
\[
\rho = \lvert\psi\rangle\langle\psi\rvert,
\]
o traço parcial produz uma matriz densidade \(2\times2\) que descreve o
estado de um único qubit como se o outro não existisse mais. Para
estados separáveis (não emaranhados), o resultado é novamente um estado
puro; já para estados emaranhados, como
\[
\lvert\psi\rangle = \frac{1}{\sqrt{2}}(\lvert 00\rangle - \lvert 11\rangle),
\]
o traço parcial produz
\[
\rho_0 = \rho_1 = \frac{1}{2} I,
\]
que é um estado misto: não aponta para nenhum ponto específico da esfera
de Bloch, pois representa incerteza total sobre o qubit isolado. Isso
ocorre porque o estado global possui correlações quânticas que não podem
ser atribuídas a nenhum dos qubits individualmente. Em resumo,
\textbf{tirar o traço} significa eliminar matematicamente a parte do
sistema que não se deseja descrever, resultando em um estado reduzido
que captura apenas a estatística observável do subsistema restante.


    % Add a bibliography block to the postdoc
    
    
    
\end{document}
